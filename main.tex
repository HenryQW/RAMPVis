% $Id: template.tex 11 2007-04-03 22:25:53Z jpeltier $

\PassOptionsToPackage{x11names}{xcolor}

\documentclass{vgtc}                          % final (conference style)
%\documentclass[review]{vgtc}                 % review
%\documentclass[widereview]{vgtc}             % wide-spaced review
%\documentclass[preprint]{vgtc}               % preprint
%\documentclass[electronic]{vgtc}             % electronic version

%% Uncomment one of the lines above depending on where your paper is
%% in the conference process. ``review'' and ``widereview'' are for review
%% submission, ``preprint'' is for pre-publication, and the final version
%% doesn't use a specific qualifier. Further, ``electronic'' includes
%% hyperreferences for more convenient online viewing.

%% Please use one of the ``review'' options in combination with the
%% assigned online id (see below) ONLY if your paper uses a double blind
%% review process. Some conferences, like IEEE Vis and InfoVis, have NOT
%% in the past.

%% Figures should be in CMYK or Grey scale format, otherwise, colour 
%% shifting may occur during the printing process.

%% These few lines make a distinction between latex and pdflatex calls and they
%% bring in essential packages for graphics and font handling.
%% Note that due to the \DeclareGraphicsExtensions{} call it is no longer necessary
%% to provide the the path and extension of a graphics file:
%% \includegraphics{diamondrule} is completely sufficient.
%%
\ifpdf%                                % if we use pdflatex
  \pdfoutput=1\relax                   % create PDFs from pdfLaTeX
  \pdfcompresslevel=9                  % PDF Compression
  \pdfoptionpdfminorversion=7          % create PDF 1.7
  \ExecuteOptions{pdftex}
  \usepackage{graphicx}                % allow us to embed graphics files
  \DeclareGraphicsExtensions{.pdf,.png,.jpg,.jpeg} % for pdflatex we expect .pdf, .png, or .jpg files
\else%                                 % else we use pure latex
  \ExecuteOptions{dvips}
  \usepackage{graphicx}                % allow us to embed graphics files
  \DeclareGraphicsExtensions{.eps}     % for pure latex we expect eps files
\fi%

%% it is recomended to use ``\autoref{sec:bla}'' instead of ``Fig.~\ref{sec:bla}''
\graphicspath{{figures/}{pictures/}{images/}{./}} % where to search for the images

\usepackage{microtype}                 % use micro-typography (slightly more compact, better to read)
\PassOptionsToPackage{warn}{textcomp}  % to address font issues with \textrightarrow
\usepackage{textcomp}                  % use better special symbols
\usepackage{mathptmx}                  % use matching math font
\usepackage{times}                     % we use Times as the main font
\renewcommand*\ttdefault{txtt}         % a nicer typewriter font
\usepackage{cite}                      % needed to automatically sort the references
\usepackage{tabu}                      % only used for the table example
\usepackage{booktabs}                  % only used for the table example
%% We encourage the use of mathptmx for consistent usage of times font
%% throughout the proceedings. However, if you encounter conflicts
%% with other math-related packages, you may want to disable it.

%% If you are submitting a paper to a conference for review with a double
%% blind reviewing process, please replace the value ``0'' below with your
%% OnlineID. Otherwise, you may safely leave it at ``0''.
\onlineid{0}

%% declare the category of your paper, only shown in review mode
\vgtccategory{Research}

%% allow for this line if you want the electronic option to work properly
\vgtcinsertpkg

%% In preprint mode you may define your own headline. If not, the default IEEE copyright message will appear in preprint mode.
%\preprinttext{To appear in an IEEE VGTC sponsored conference.}

%% This adds a link to the version of the paper on IEEEXplore
%% Uncomment this line when you produce a preprint version of the article 
%% after the article receives a DOI for the paper from IEEE
%\ieeedoi{xx.xxxx/TVCG.201x.xxxxxxx}

\usepackage[nameinlink]{cleveref}
\usepackage[nolist,nohyperlinks]{acronym}

\newcommand{\bobgraph}[1]{%
    \par\medskip
    \noindent\textbf{#1}%
    \par\medskip
}

\newcommand{\newtext}[1]{%
    \textcolor{red}{#1}%
}

\acrodef{VIS}{Data Visualization and Visual Analytics}
\acrodef{SCRC}{The Scottish COVID-19 Response Consortium}
\acrodef{ABC-SMC}{Approximate Bayesian Computation Sequential Monte Carlo}

%% Paper title.

\title{EnsembleVis View and Volunteers: A Retrospective and Early History}

%% This is how authors are specified in the conference style

%% Author and Affiliation (single author).
%%\author{Roy G. Biv\thanks{e-mail: roy.g.biv@aol.com}}
%%\affiliation{\scriptsize Allied Widgets Research}

%% Author and Affiliation (multiple authors with single affiliations).
%%\author{Roy G. Biv\thanks{e-mail: roy.g.biv@aol.com} %
%%\and Ed Grimley\thanks{e-mail:ed.grimley@aol.com} %
%%\and Martha Stewart\thanks{e-mail:martha.stewart@marthastewart.com}}
%%\affiliation{\scriptsize Martha Stewart Enterprises \\ Microsoft Research}

%% Author and Affiliation (multiple authors with multiple affiliations)
\author{Qiru Wang\thanks{e-mail: qiru.wang1@nottingham.ac.uk}\\ %
        \scriptsize University of Nottingham %
\and Robert S. Laramee\thanks{e-mail: robert.laramee@nottingham.ac.uk}\\ %
     \scriptsize University of Nottingham}

%% A teaser figure can be included as follows
% \teaser{
%   \centering
%   \includegraphics[width=\linewidth]{overview.png}
%   \caption{An overview of EnsembleVis, an interactive dashboard designed for visualizing the input parameters and outcomes of an Approximate Bayesian Computation Sequential Monte Carlo (ABC-SMC) inference model used for analyzing COVID-19 data collected during the first wave of the outbreak in Scotland \cite{2020Covid19}. Its main purpose is to facilitate a better understanding of the complex dynamics of the pandemic by presenting the relationships between different sets of simulation input parameters and the resulting outcomes in a clear and intuitive manner. A detailed description of the dashboard can be found in \Cref{sec:EnsembleVis}.
%   }
%   \label{fig:teaser}
% }

%% Abstract section.
\abstract{
  This paper offers a retrospective view on the early development stages of EnsembleVis, a visualization dashboard specifically crafted to support modelers in interpreting a simulation model utilized to forecast COVID-19 trends. The voluntary effort behind this dashboard was contributed collaboratively with the Scottish COVID-19 Response Consortium (SCRC) with the objective of facilitating an enhanced comprehension of the pandemic's complex dynamics via modeling of a real-world dataset collected by NHS Scotland during the first wave of the outbreak. This retrospective chronicles the design and development journey of the system, supplemented by feedback from domain experts, all taking place amidst the exceptional circumstances of an unprecedented pandemic. The outcome of this volunteer work is a streamlined relationship discovery process between various sets of simulation input parameters and their respective outcomes, that leverages the power of information visualization and visual analytics (VIS). We hope that this retrospective will serve as an insightful resource for future effort in visualization for pandemic and emergency responses, and promote mutually beneficial engagement between scientific communities.
} % end of abstract

%% ACM Computing Classification System (CCS). 
%% See <http://www.acm.org/about/class> for details.
%% We recommend the 2012 system <http://www.acm.org/about/class/class/2012>
%% For the 2012 system use the ``\CCScatTwelve'' which command takes four arguments.
%% The 1998 system <http://www.acm.org/about/class/class/2012> is still possible
%% For the 1998 system use the ``\CCScat'' which command takes four arguments.
%% In both cases the last two arguments (1998) or last three (2012) can be empty.

\CCScatlist{
  \CCScatTwelve{Visual Analytics}{Information Visualization}{Emergency Response}{Visual Design}
}

%\CCScatlist{
  %\CCScat{H.5.2}{User Interfaces}{User Interfaces}{Graphical user interfaces (GUI)}{};
  %\CCScat{H.5.m}{Information Interfaces and Presentation}{Miscellaneous}{}{}
%}

%% Copyright space is enabled by default as required by guidelines.
%% It is disabled by the 'review' option or via the following command:
% \nocopyrightspace

%%%%%%%%%%%%%%%%%%%%%%%%%%%%%%%%%%%%%%%%%%%%%%%%%%%%%%%%%%%%%%%%
%%%%%%%%%%%%%%%%%%%%%% START OF THE PAPER %%%%%%%%%%%%%%%%%%%%%%
%%%%%%%%%%%%%%%%%%%%%%%%%%%%%%%%%%%%%%%%%%%%%%%%%%%%%%%%%%%%%%%%%

\begin{document}

%% The ``\maketitle'' command must be the first command after the
%% ``\begin{document}'' command. It prepares and prints the title block.

%% the only exception to this rule is the \firstsection command
% \firstsection{Introduction}

\maketitle

\section{Introduction and Motivation}
\label{sec:intro}

The Scottish COVID-19 Response Consortium (SCRC) \cite{2020University}, in collaboration with the Royal Society's call to action in March 2020, has taken a proactive approach to address the need for enhanced epidemiological models of COVID-19 transmission.
This joint volunteer effort, known as Rapid Assistance in Modeling the Pandemic (RAMP) \cite{2020Rapid}, aims to foster a deeper understanding of the consequences associated with various exit strategies from lockdown measures.
Moreover, this consortium attracted the involvement of distinguished scientists and experts from diverse organizations both within the United Kingdom and abroad, thus augmenting the collective knowledge base and ensuring comprehensive expertise in specialized domains.

RAMPVis \cite{2020Visualization} is a group of researchers specialized in Data Visualization and Visual Analytics (abbreviated as VIS).
This group voluntarily came forward to contribute its specialized skills and knowledge in order to provide valuable support to the SCRC modelers.
The term \textit{modelers} used here refers to the SCRC researchers who were actively engaged in the development of epidemiological models in the SCRC.
% revision 
This target user group predominantly includes experts in domains such as mathematics, statistics, and epidemiology.

Serving as the volunteer team responsible for providing visualization support to one of the epidemiological models developed by the SCRC modelers \cite{chen2022RAMPVIS}, our main objective is to provide VIS researchers and practitioners with valuable insights gained from our research and development (R\&D) activities conducted during the COVID-19 pandemic.
In an effort to predict the potential impact of diverse interventions, modelers have actively utilized COVID-19 data, employing a method known as Uncertainty Quantification (UQ).
This process seeks to measure uncertainties through the application of mathematical models and simulations.
However, modelers are faced with significant challenges, including the aspects of expert elicitation and effective communication.
In other words, there is a need for software engineering effort coupled with visualization to provide support for the validation and verification tests, and to create efficient workflows between modelers and researchers from other disciplines \cite{ackland2022Royal}.

\begin{figure}[tb!]
    \centering
    \includegraphics[width=0.65\linewidth]{venn.png}
    \caption{The organization of researchers from the SCRC and RAMPVis. The SCRC modeling team is responsible for developing the epidemiological models leveraging different modeling techniques. The RAMPVis team provides visualization support to the SCRC modeling team, by establishing four VIS volunteer teams who work on the actual development under the guidance of the RAMPVis team.
    }
    \label{fig:venn}

\end{figure}

In addressing these hurdles, \ac{VIS} emerge as a potent tool, offering the capacity to significantly enhance and streamline their collaborative workflows \cite{swallow2022Challenges}.
While our work may not have showcased the state-of-the-art VIS techniques, it effectively delivered rapid and practical VIS support to the modelers during an exceptional and demanding time.

Our contribution is an early history of our volunteer response from a software engineering and visualization perspective.
We present the earliest stages of the visualization dashboard, EnsembleDashVis, developed during the pandemic, aiming to assist the modelers in interpreting an \ac{ABC-SMC} inference model that they have developed using COVID-19 data collected during the first wave of the outbreak in Scotland \cite{scrc2020Covid19}. 
Much of this effort and the reasoning behind this volunteer work was never documented.

% revision
\noindent
\textbf{Unconventional Software Development:}  
The visualization software we developed in this project was developed under unconventional and unprecedented circumstances.
Some of the unusual properties of this software project were the amount of uncertainty as the project started.
The following aspects were unknown at the project start:
\begin{itemize}[itemsep=0pt,topsep=0pt]
    \item An unknown a priori requirements specification: We did not know what the user requirements and expectations were.
    \item An unknown project team: The members of the project team were unknown.
    We only knew the leader of the visualization team, Prof Min Chen. 
    In addition, the project team was dynamic with new members joining throughout.
    \item Unknown data characteristics: We did not know what the simulation data was at the start of the project.
    \item An unfamiliar work environment: The collective work environment landscape shifted to a work-at-home model which was new to the team at the time.
\end{itemize}
While arguably, these characteristics could describe other software engineering projects, we believe that the uncertainty in this particular case was unusually high.
All aspects of this project had the feel of laying down the tracks as the train was running.

\section{Background and Related Work}

\ac{VIS} has been widely utilized in critical applications such as emergency responses and healthcare, assisting public officials and decision-makers in understanding intricate datasets and extracting useful, actionable insights from them \cite{dusse2016Information}. 
\ac{VIS} has also played a prominent role in disseminating COVID-19 information through various media channels, it has significantly contributed to enabling more efficient and clearer public communication, facilitating a broader understanding of the crisis \cite{johnshopkinsuniversityCOVID19}.

In our work, our primary objective was to extend support through \ac{VIS} to two distinct user groups. Firstly, the modelers, who could significantly benefit from VIS in comprehending their models more effectively and fine-tuning them accordingly. Secondly, to the epidemiologists, whom VIS could assist in interpreting the outcomes of these computational models. Our work is included in multiple publications \cite{chen2022RAMPVIS,dykes2022Visualizationb,khan2022Propagating,khan2022Rapid,rydow2023RAMPVIS}, where it functioned as the preliminary VIS prototype, shaping a portion of their respective studies.

\bobgraph{VIS for Emergency Response}

Previously as co-authors, we have detailed the related work focusing the use of \ac{VIS} in emergency response, refer to the related work section in Chen et al. \cite{chen2022RAMPVIS}. The aforementioned literature review laid the foundation and was conducted prior to the development of our current study in 2020.

Maciejewski et al. \cite{maciejewski2011Pandemica} develop a VIS toolkit for analyzing the effect of decision measures enforced during a simulated pandemic, the tool was later utilized by Indiana State Department of Health during an H1N1 (swine flu) outbreak.
Ribicic et al. \cite{ribicic2012Sketching} leverage VIS with the intention of delivering real-time feedback derived from flood simulations to non-expert users, while Konev et al. \cite{konev2014Run} use VIS to support decision-making in flooding scenarios.

Jeitler et al. \cite{jeitler2019RescueMark} use VIS to analyze social media data to aid rescue teams, specifically in terms of optimally allocating resources during emergency response situations. Similarly, Nguyen and Dang \cite{nguyen2019EQSA} harness social media data, paired with VIS, to facilitate and enhance post-earthquake resource allocation and rescue effort.

In contrast to the majority of previous studies mentioned here that typically focus on preparing for future emergencies, our work was undertaken during the COVID-19 pandemic as a rapid response to an ongoing emergency.

\bobgraph{VIS for COVID-19 Data Modeling}

In the rest of the section, we focus on the use of \ac{VIS} in aiding the computational modeling of COVID-19 data. These studies were not published or available to us during the development of our work. In fact the use of VIS in epidemiological modeling was rare, both modelers and epidemiologists might be unaware that they had such a potent instrument readily available \cite{chen2022RAMPVIS}.

He et al. \cite{he2020SEIR} developed an SEIR (Susceptible, Exposed, Infected, and Recovered) model for spread prediction by leveraging COVID-19 data obtained from the Hubei province in China. They employed a variety of 2D plots for estimating the parameters of the model and interpreting the outcomes yielded by the model. Godio et al. \cite{godio2020SEIR} took the same approach in their development of an SEIR model for the Lombardy region in Italy.

The IHME COVID-19 Forecasting Team \cite{ihmecovid-19forecastingteam2021Modeling} took the application of data visualization (VIS) a step further in their development of SEIR model for accessing social distance mandates, they extend the use of \ac{VIS} to include choropleth and violin plots, and small multiples for 2D plots.

Chinazzi et al. \cite{chinazzi2020Effect} developed a model for simulating the effectiveness of international travel restrictions in containing the spread of COVID-19. Besides the employment of 2D plots for refining their models, they also utilized a range of geospatial visualizations. This allowed them to more effectively interpret the results generated by their models. The use of geospatial visualizations is also adopted by Alvarez Castro and Ford \cite{alvarezcastro20213D} in their development of a model for analyzing the transmission in a UK university campus.

Contrary to these studies that showcase the efficacy of \ac{VIS} in supporting computational modeling of COVID-19 data with a primary focus on the model's development, as they are formulated by the modelers, our study takes a different approach. We focus our attention on refining \ac{VIS} as a potent tool that can significantly enhance the computational modeling of COVID-19 data, all viewed through the unique lens of a \ac{VIS} practitioner.


\section{Data Description}
\label{sec:Data}

\begin{table}[htb!]
    \centering
    \caption{16 input parameters used for the ABC-SMC inference model.}
    \label{tab:16params}
    \resizebox{\columnwidth}{!}{%
        \begin{tabular}{|l|c|p{9.25cm}|}
            \hline
            \textbf{ID} & \textbf{Compart} & \textbf{Description}                                                                                        \\
            \hline
            S           & 0                & Number of susceptible individuals (not infected).                                                           \\
            \hline
            E           & 1                & Number of infected individuals but not yet infectious (exposed).                                            \\
            \hline
            E\_t        & 2                & Number of exposed individuals and tested positive.                                                          \\
            \hline
            I\_p        & 3                & Number of infected and infectious symptomatic individuals but at pre-clinical stage (show yet no symptoms). \\
            \hline
            I\_t        & 4                & Number of tested positive individuals that are infectious.                                                  \\
            \hline
            I1          & 5                & Number of infected and infectious asymptomatic individuals: first stage.                                    \\
            \hline
            I2          & 6                & Number of infected and infectious asymptomatic individuals: second stage.                                   \\
            \hline
            I3          & 7                & Number of infected and infectious asymptomatic individuals: third stage.                                    \\
            \hline
            I4          & 8                & Number of infected and infectious asymptomatic individuals: last stage.                                     \\
            \hline
            I\_s1       & 9                & Number of infected and infectious symptomatic individuals: first stage.                                     \\
            \hline
            I\_s2       & 10               & Number of infected and infectious symptomatic individuals: second stage.                                    \\
            \hline
            I\_s3       & 11               & Number of infected and infectious symptomatic individuals: third stage.                                     \\
            \hline
            I\_s4       & 12               & Number of infected and infectious symptomatic individuals: last stage.                                      \\
            \hline
            H           & 13               & Number of infected individuals that are hospitalized.                                                       \\
            \hline
            R           & 14               & Number of infected individuals that have recovered from the infection.                                      \\
            \hline
            D           & 15               & Number of deceased individuals due to the disease.                                                          \\
            \hline
        \end{tabular}%
    }
\end{table}

\begin{table}[htb!]
    \centering
    \caption{Description of output parameters.}
    \label{tab:outparams}
    \resizebox{\columnwidth}{!}{%
        \begin{tabular}{|l|p{10cm}|}
            \hline
            \textbf{Name} & \textbf{Description}                                                                    \\ \hline
            iterID        & The simulation number.                                                                   \\ \hline
            age\_group    & The age group of the population.                                                         \\ \hline
            comparts      & The epidemiological compartment number or the parameter space (See \Cref{tab:16params}). \\ \hline
            value         & The population of that compartment at the end of the simulation run.                     \\ \hline
        \end{tabular}%
    }
\end{table}

The data used in our work includes simulation parameters and outcomes from an ABC-SMC inference model \cite{toni2008Approximate} built by a group of modelers from University of Edinburgh, University of Exeter, University of Glasgow, and London School of Hygiene \& Tropical Medicine. The underlying pandemic data was collected during the first wave of the outbreak in Scotland. The model was built to analyze the data and infer the parameters of the model that best fit the data. The model was run 1,000 times with different random seeds to generate a set of 160 parameter sets. The model has 16 input parameters (see \Cref{tab:16params}) and 4 output parameters (see \Cref{tab:outparams}). The resulting output file is a CSV file that is over 6GB in size.

It is worth mentioning that after plotting the output data using a line chart, an error was immediately spotted, see \Cref{fig:teaser}D, where an unusual spike at day 20 can be observed. The modelers were notified and the bug was fixed. However, the rectified output file was never made available to us.


\section{Demers Cartogram with Rivers}


\section{Domain Expert Feedback}
\label{sec:feedback}
In this section, we share the invaluable feedback collected from the modelers. 
Meeting \#6 and \#7 were held prior to the conclusion of our development, serving as an iterative process of refinement aimed at validating and improving our visual designs while ensuring their relevance and utility to domain experts.
Meeting \#8 was held after the conclusion of our development, functioning as a means to gather feedback on our work and to identify potential future work.
Here, we present some of the original quotes collected from the domain experts during these meetings.

\bobgraph{Domain Expert \#1 - Professor in Statistics, Durham University}
% Ian Vernon

During our live demonstration, the domain expert appreciated the interactions provided by the visual designs in depicting the relative importance of different input parameters on the model's predictions.
\textit{``The visualizations are able to show how important a particular input is for a particular output.''
    }

In addition, the ability to present the link between the input and output parameters visually.
\textit{``The real interesting game here is the connection techniques to understand the relations between the input and output.''
}

The linked visual designs also potentially enable the domain expert to identify ineffective parameter combinations.
\textit{``The different configurations is the sort of history of calibration and by looking at those visualizations you can start saying certain combinations may not be useful.''
}

Furthermore, the domain expert also expressed interest in the potential of our visual designs to aid in the identification of model discrepancy, when the observational data becomes available.
\textit{``The visualization would be helpful for us to identify model discrepancy when we eventually plot the observational data.''
}

\bobgraph{Domain Expert \#2 - Professor in Statistics, the University of Exeter}
% Peter Challenor

The domain expert was pleased by the visual designs' ability to provide the potential to filter redundant input parameter configurations, enabling users to concentrate on the most influential configurations.
\textit{``For particular input configurations after filtering, the visualization shows some of the input parameters can be ignored, which reduces the dimensionality of the problem, and we can focus on the important parameters.''
}

\bobgraph{Domain Expert \#3 - Assistant Professor in Statistics, the University of Glasgow}
% Ben Swallow
The domain expert praised the visual designs' ability to provide a clear overview of the input parameters and their distributions. This enables them to quickly identify possibility adjustments they can make to their input parameters, as well as to identify the most influential parameters.
\textit{
``The table view is really useful in showing how close those input parameters are to the threshold, which is very useful for us to understand the affordability.''
}

The domain expert also noted that some overlapping distributions can be ruled out quickly via the interactivity provided by our visual designs, this enables them to eliminate unnecessary complexities and increase the overall efficiency of their model.
\textit{
``It's fairly obvious that some of the parameters can be ruled out quite quickly, including some overlapping distributions.''
}

An avenue for future work, as unanimously identified by all three domain experts, involves integrating new visual designs to effectively render and compare observational data against model predictions.


\section{Limitations}

Due to the impact of the pandemic, the project was conducted in a fully virtual manner, with all meetings and discussions taking place online, between a very large group of researchers from different disciplines. This has resulted in a number of limitations, which we will discuss in this section.

\subsection{Lack of Novel and Advanced Visual Designs}
Operating under a time constraint, the primary objective of our project centered on offering immediate visualization assistance to the modelers.
Thus, we were unable to explore the inclusion of innovative and advanced visual design approaches.
Instead, we integrated a series of classic visualizations, such as line charts and scatterplots.
These are visual elements commonly leveraged by modelers and epidemiologists in their day-to-day research.
Interestingly, the modelers welcomed the introduction of a less conventional (to them) visualization technique: parallel coordinates.
They had never before employed this visual design, and its introduction proved beneficial to their research.
Consequently, they expressed a desire for the incorporation of an additional parallel coordinates to assist in the visualization of model outcomes.

We believe that this is a testament to the effectiveness of advanced visual designs in enhancing the modelers' understanding of their models, this signals the possibility for future inclusion of more sophisticated visual designs.

\subsection{Lack of Proper Requirement Gathering}
We were unable to meet with the modelers and epidemiologists until the very last meeting. Instead, we had to rely on email correspondence, which was arguably not as effective as face-to-face or even virtual meetings.

In a traditional software engineering project, the requirements are gathered through a series of meetings and discussions with the end users. This did not happen in our case.

This resulted in a lack of proper requirement gathering, which in turn led to a number of issues during the development process.
For example, the modelers made ad-hoc requests to incorporate different visualizations at different stages of the project, resulting in unexpected changes on the development side.
This could have been avoided if we had a better understanding of their requirements from the beginning.

\subsection{Dynamic Group Membership}

The group membership was dynamic, with researchers joining and leaving the group at different stages of the project.
This resulted in a lack of continuity, as newcomers had to spend time to familiarize themselves with the project.
Furthermore, members came from different disciplines, with different levels of expertise in visualization.
This has resulted in a lack of consistency in the development process, as different members have different ideas on how to implement visualizations.
The responsibility of each member, apart from the only developer in the group, was not clearly defined.

\subsection{Unclarified Project Direction}

The exact direction of the project was not clearly defined from the beginning.
Many details remained unknown to us during the development process, such as the exact purpose of the visualization, the target audience, and the end product. 
Consequently, the final product suffered from a non-ideal utilization of screen-space, as more visualizations were requested to be added, the implementation of a multi-screen display design or collapsible views became time-constrained and unachievable.


\section{Conclusions}

In this paper, we have presented the stories behind the development of EnsembleVis, an interactive dashboard designed for visualizing the input parameters and outcomes of an ABC-smc inference model used for analyzing COVID-19 data collected during the first wave of the outbreak in Scotland.

Given the multitude of uncertainties and challenges during this exceptional period, a considerable amount of information was unavailable to us during the development process. It was only through the Scottish COVID-19 Response Consortium Stakeholder Report \cite{abdalla2021Scottish}, published in late 2021, and various publications \cite{chen2022RAMPVIS,dykes2022Visualizationb,khan2022Propagating,rydow2023RAMPVIS} that unveiled the remarkable endeavors undertaken by other volunteer teams, that we gained additional insights and details.

\begin{figure}
    \centering
    \includegraphics[width=\linewidth]{rampvis.png}
    \caption{EnsembleVis has undergone extensive development by multiple UK institutions and is currently maintained by the Oxford e-Research Centre at the University of Oxford, serving as a vital element of the RAMPVIS infrastructure. It is well-prepared to offer rapid and invaluable visualization support for future emergency responses. Image courtesy of Rydow et al. \cite{rydow2023RAMPVIS}.
    }
    \label{fig:rampvis}

\end{figure}

We hope that our experience serves as a valuable source of insights on how VIS research and techniques can play a crucial role in emergency response initiatives and aid in effectively preparing for future emergencies.

%% if specified like this the section will be committed in review mode
% \acknowledgments{
% The authors wish to thank A, B, and C. This work was supported in part by
% a grant from XYZ.}

% \bibliographystyle{abbrv}
% \bibliographystyle{abbrv-doi}
%\bibliographystyle{abbrv-doi-narrow}
\bibliographystyle{abbrv-doi-hyperref}
%\bibliographystyle{abbrv-doi-hyperref-narrow}

\bibliography{RAMPVIS}
\end{document}
