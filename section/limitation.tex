\section{Limitations}

Due to the impact of the pandemic, the project was conducted in a fully virtual manner, with all meetings and discussions taking place online, between a large group of researchers from different disciplines and different institutions.
In total, 33 VIS researchers and 7 modelers were involved in this volunteer work.
The development was ad-hoc in some ways due to the unprecedented nature of the pandemic.
This resulted in a number of limitations, which we will discuss in this section.

\vspace*{3mm}
\noindent
\textbf{Lack of Novel and Advanced Visual Designs:}
Operating under a time constraint, the primary objective of our project centered on offering immediate visual analysis assistance to the modelers.
Thus, we were unable to explore the inclusion of innovative and advanced visual design approaches.
Instead, we integrated a series of classic views, such as line charts and scatterplots.
These are visual designs commonly leveraged by modelers in their day-to-day research.
Interestingly, the modelers welcomed the introduction of a less conventional (to them) visualization technique: parallel coordinates.
They had never before employed this, and its introduction proved beneficial to their research.
Consequently, they expressed a desire for the incorporation of an additional parallel coordinates to assist in the visualization of model outcomes.

We believe that this is a testament to the effectiveness of advanced visual designs in enhancing the modelers' understanding of their models, this signals the possibility for future inclusion of more sophisticated visual designs.

\vspace*{3mm}
\noindent
\textbf{Lack of Formal Requirements Gathering:}
We were unable to meet with the modelers until a particularly late stage. Instead, we had to rely on email correspondence, which was arguably not as effective as face-to-face or even virtual meetings.
In a traditional software engineering project, requirements are gathered through a series of meetings and discussions with end users. This did not occur in our case.

This resulted in a lack of proper requirement gathering, which in turn led to a number of challenges during the development process.
For example, the modelers made ad-hoc requests to incorporate different views at different stages of the project, resulting in unexpected changes on the development side.
This could have been avoided if we had a better understanding of their requirements from the beginning.

\vspace*{3mm}
\noindent
\textbf{Dynamic Group Membership:}
The group membership was dynamic, with researchers joining and leaving the group at different stages of the project.
This introduced some lack of continuity, as newcomers had to spend time to familiarize themselves with the project.
Furthermore, members came from different disciplines, with different levels of expertise in visualization.
This has resulted in a lack of consistency in the development process, as different members have different ideas on how to implement the views.
The responsibility of each member, apart from the only developer in the group, was not clearly defined.

\vspace*{3mm}
\noindent
\textbf{Uncertain Project Direction:}
The exact direction of the project was not clearly defined from the beginning.
Many details remained unknown to us during the development process, such as the exact purpose of the visualization, the target audience, and the end product.
Consequently, the final product suffered from a non-ideal utilization of screen-space, as more views were requested to be added, the implementation of a multi-screen display design or collapsable views became time-constrained and unachievable.

% Revision
\vspace*{3mm}
\noindent
\textbf{Other Technical Limitations:} 
Some additional technical limitations include:
\begin{itemize}[itemsep=0pt,topsep=0pt]
    \item Real-time updating: Coupling the simulation with the visual rendering directly would have been very beneficial to the project, e.g., computational steering.
    \item Standardization: Standardization of the data format would be beneficial to all participants.
    \item Interpretability: A more formal evaluation of how interpretable our visual representations are would be beneficial, e.g., presenting complex epidemiological concepts in a clear and understandable manner to a wider audience.
\end{itemize}