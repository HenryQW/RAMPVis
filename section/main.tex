\section{EnsembleVis}
\label{sec:EnsembleVis}

This section presents the history and development behind our fully virtual collaboration between researchers from multiple UK institutions.
\newtext{Being one of the six VIS volunteer teams, we received guidance from the RAMPVis team via regular virtual meetings. The RAMPVis team communicated with the SCRC modeling team regularly, and provided us with the necessary information and data.}
We chronicle the development of different views of the data, the order in which they were introduced and the reasons and motivations at the time.
In 2020 we were all in an unprecedented and unfamiliar situation, thus, some of our decisions were ad-hoc.

\bobgraph{Meeting \#1 - July 2020}

\label{subsec:InitialMeeting}
On 27 July 2020, amid the UK's first national lockdown and stricter measures imposed by local authorities, we convened the initial virtual meeting with VIS researchers from King's College London, Loughborough University, Swansea University, University of Nottingham, University of Warwick, and University of Oxford.

During the meeting, we received an overview of the SCRC and the responsibilities of the visualization volunteer team.
Our assigned task was to create visual interfaces for the model, for the purpose of enabling the modelers to analyze the outcomes of the model.

Following the initial meeting, we engaged in email correspondence with the modelers to delve into the visualization requirements. The modelers shared a comprehensive list of parameters and model outcomes, along with the corresponding outcome data.

\bobgraph{Commit \#1 - Sep 2020}

We proceeded to create an initial prototype of the visualization, which was subsequently reviewed by the modelers.
Incorporating their input, we refined the prototype during our weekly internal discussions.
On 14 Sep 2020, England introduced the `rule of six', which banned any gatherings above six.
On the same day, we made our first commit to a GitHub repository, signifying the commencement of our development.
At the same time, we began preprocessing the data.
A week after the initial commit, the UK witnessed the implementation of additional restrictions, such as mandatory work from home and a 10pm curfew.

\bobgraph{Meeting \#2, View \#1 - Nov 2020}

On 5 Nov 2020, the first day of the second national lockdown in the UK, we completed the first view of the simulated input parameters, a parallel coordinates plot, see \Cref{fig:pc1}.
\newtext{We chose to use a parallel coordinates plot as it is a common technique for visualizing multivariate data, and is particularly useful to explore relationships and patterns across multiple input parameters.}
This followed by the second meeting with the RAMPVis team from other institutions, where we received feedback on the first view, on 6 Nov 2020.

\begin{figure}[tb!]
    \centering
    \includegraphics[width=\linewidth]{pc1.png}
    \caption{The first visual design, a parallel coordinates plot depicting all 160 input configurations of the model, was completed on 5 Nov 2020. The table below shows the configuration details.
    }
    \label{fig:pc1}

\end{figure}

\bobgraph{Meeting \#3, View \#2 - Nov 2020}

On 11 Nov 2020, the group convened for the third meeting, where we received further feedback from the RAMPVis team on the parallel coordinates. As per the modelers' requests conveyed via email, we incorporated a line chart to depict the model outcomes, see \Cref{fig:1st-line}.
\newtext{Line chart and other classic visual designs are widely used by the modelers, they are familiar with these designs and can easily interpret the results.}

\begin{figure}[tb!]
    \centering
    \includegraphics[width=0.7\linewidth]{1st-line.png}
    \caption{A line chart depicting the model outcomes.
    The x-axis of the chart corresponds to the number of days since the first date in the Scottish dataset, while the y-axis represents the population.
    To differentiate between different population categories, a color map was incorporated: \textcolor{DodgerBlue1}{susceptible}, \textcolor{Chocolate1}{exposed}, \textcolor{Green4}{hospitalized}, \textcolor{red}{recovered}, \textcolor{DarkOrchid1}{death}, \textcolor{LightSalmon4}{asymptomatic}, and \textcolor{HotPink1}{symptomatic}.
    }
    \label{fig:1st-line}

\end{figure}

\bobgraph{Meeting \#4, View \#3 - Nov 2020}

On 25 Nov 2020, the group convened for the fourth meeting, held just a day after the announcement of the gathering rules for Christmas in the UK.
During the meeting, we received feedback from the RAMPVis team on the new view of the input parameters, a table with glyphs, see \Cref{fig:table-view}. We incorporated this table view featuring glyphs to depict all 160 input parameter configurations, following discussions with the modelers.
\newtext{The table views enables the modelers to sort configurations by individual input parameters. Each parameter value is symbolized by a bar glyph, with the color of the glyph corresponding to its deviation from the group's average value.}

The table view provides the functionality to sort the parameters according to their values and can be dynamically updated by brushing the parallel coordinates plot for the input parameters in \Cref{fig:pc1}. The line chart in \Cref{fig:1st-line} can be quickly updated to display the corresponding model outcomes by clicking on the configuration index in the table view.

\begin{figure}[tb!]
    \centering
    \includegraphics[width=\linewidth]{table.png}
    \caption{The table view depicting all 160 input parameter configurations.
    The view enables the modelers to sort parameter values and identify interesting configurations.
    Upon clicking on the index of a configuration, the line chart in \Cref{fig:1st-line} is updated to display the corresponding model outcomes.
    }
    \label{fig:table-view}

\end{figure}

\bobgraph{Meeting \#5 - Dec 2020}

On 9 Dec 2020, a week after the end of the second national lockdown in the UK, with England facing a stricter three-tier restriction policy, the group convened for the fifth meeting.
At this point, we still had not met with the modelers, all communications and discussions took place through emails.
The RAMPVis team decided to organize a meeting with the modelers, to present our work for feedback.

\bobgraph{Meeting \#6, View \#4, Feedback \#1 - Dec 2020}

On 10 Dec 2020, a week after the second UK national lockdown, we finally met with modelers from Durham University, the University of Edinburgh, the University of Exeter, the University of Glasgow, the London School of Hygiene \& Tropical Medicine, for the first time.
In contrast to sharing screenshots via email and deploying a website with a live view of our development (which they might not have been proficient in using), we delivered a live presentation, fielding numerous questions.
The modelers were extremely pleased with the visualization, \newtext{and a list of ad-hoc requirements was provided. Additionally, we gathered insightful feedback which we elaborate on in detail in \Cref{sec:feedback}}.
\begin{enumerate}
    \item The modelers found the parallel coordinates plot very useful, and requested the incorporation of another one for the model outcomes. \newtext{Given that the outcome data mirrors the input in a multivariate format, employing a parallel coordinates plot could potentially be useful.} We implemented this as shown in \Cref{fig:pc2}.
    \item The modelers requested all the simulation results be displayed in the line chart, with the current one highlighted. \newtext{This resembles their usual workflow of analyzing multiple simulation outcomes.} We implemented this as shown in \Cref{fig:1st-line}.
    \item The modelers requested the incorporation of a scatterplot to visualize the model outcomes, specifically a Principal Component Analysis (PCA) result obtained from another VIS volunteer team. \newtext{The motivation behind is to reduce the dimensionality and identify key parameters.} We implemented this as shown in \Cref{fig:pca}.
\end{enumerate}

Furthermore, we received the exciting news that initial funding had been successfully secured \cite{engineering&physicalsciencesresearchcouncil2021RAMP}, leading to the transition of our voluntary work to a team of paid developers, who would continue with further implementation of the project.

\begin{figure}[tb!]
    \centering
    \includegraphics[width=\linewidth]{pc2.png}
    \caption{A parallel coordinates plot depicting the model outcomes by age group. As requested by the modelers, the mean value of 1,000 simulations generated by each input configuration, as well as for each age group, was computed and rendered here. 
    }
    \label{fig:pc2}

\end{figure}

\begin{figure}[tb!]
    \centering
    \includegraphics[width=0.6\linewidth]{pca.png}
    \caption{A scatterplot depicting the PCA outcome from another VIS volunteer group, was added upon request by the modelers. Upon brushing, the selected configurations are highlighted in the table view in \Cref{fig:table-view}.
    }
    \label{fig:pca}

\end{figure}

\bobgraph{Meeting \#7, Feedback \#2 - Mar 2021}
On 25 Mar 2021, the UK was in the process of cautiously lifting its third national lockdown, the `rule of two' was still in place.
The group convened for the seventh meeting, where we received further feedback from the modeling team on our implementation.
We detail the feedback in detail in \Cref{sec:feedback}.

\bobgraph{Last Commit - Apr 2021}

By 28 Apr 2021, more restrictive measures were abolished, although the prohibition on mixing between households was still in effect.
On this day, we made our last commit to the GitHub repository, this act signified the completion of our volunteer work, as we had smoothly transitioned all tasks to a team of paid developers.

During the entire development process, our meetings were exclusively conducted virtually, and our communication relied heavily on email correspondence.
Despite the lack of in-person interactions, we successfully met the modelers' requirements and delivered a \ac{VIS} solution that received very positive feedback from the SCRC modeling team.

\bobgraph{Meeting \#8, Feedback \#3 - May 2021}
On 19 May 2021, the UK was seeing the light at the end of the Covid tunnel, weddings and funerals were still restricted to 30 people, and indoor gathering over two households was still banned.
The group convened for the eighth and the final meeting.
During this final meeting where a modeler joined and gave us some in-depth feedback on the influence our work had on their modeling process, as well as suggesting potential improvements.
We detail the feedback in detail in \Cref{sec:feedback}.
