\section{Introduction and Motivation}
\label{sec:intro}

The Scottish COVID-19 Response Consortium (SCRC) \cite{2020University}, in collaboration with the Royal Society's call to action in March 2020, has taken a proactive approach in addressing the need for enhanced epidemiological models of COVID-19 transmission.
This joint effort, known as Rapid Assistance in Modelling the Pandemic (RAMP) \cite{2020Rapid}, aims to foster a deeper comprehension of the consequences associated with various exit strategies from lockdown measures.
Moreover, this consortium has attracted the involvement of distinguished scientists and experts from diverse organizations both within the United Kingdom and abroad, thus augmenting the collective knowledge base and ensuring comprehensive expertise in specialized domains.

RAMPVis \cite{2020Visualization} is a group of volunteers specialized in Data Visualization and Visual Analytics (abbreviated as VIS).
This group has willingly come forward to lend their specialized skills and knowledge in order to provide invaluable support to the consortium's modeling scientists and epidemiologists.

Serving as the volunteer team responsible for providing visualization support to one of the six epidemiological models developed by the SCRC modellers \cite{chen2022RAMPVIS}, our primary aim is to bestow VIS researchers and practitioners with valuable insights gleaned from our research and development (R\&D) activities conducted during the COVID-19 pandemic.
While our work may not have showcased the state-of-the-art VIS techniques, it effectively delivered rapid and practical VIS support to the modellers during an exceptional and demanding time.
The exemplification of a fully-virtual collaboration among researchers from various UK institutions epitomizes the spirit of interdisciplinary cooperation in the fight against the pandemic.


