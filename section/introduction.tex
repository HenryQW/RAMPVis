\section{Introduction and Motivation}
\label{sec:intro}

The Scottish COVID-19 Response Consortium (SCRC) \cite{2020University}, in collaboration with the Royal Society's call to action in March 2020, has taken a proactive approach to address the need for enhanced epidemiological models of COVID-19 transmission.
This joint volunteer effort, known as Rapid Assistance in Modeling the Pandemic (RAMP) \cite{2020Rapid}, aims to foster a deeper understanding of the consequences associated with various exit strategies from lockdown measures.
Moreover, this consortium attracted the involvement of distinguished scientists and experts from diverse organizations both within the United Kingdom and abroad, thus augmenting the collective knowledge base and ensuring comprehensive expertise in specialized domains.

RAMPVis \cite{2020Visualization} is a group of researchers specialized in Data Visualization and Visual Analytics (abbreviated as VIS).
This group voluntarily came forward to contribute its specialized skills and knowledge in order to provide valuable support to the SCRC modelers.
The term \textit{modelers} used here refers to the SCRC researchers who were actively engaged in the development of epidemiological models in the SCRC.
% revision 
This target user group predominantly includes experts in domains such as mathematics, statistics, and epidemiology.

Serving as the volunteer team responsible for providing visualization support to one of the epidemiological models developed by the SCRC modelers \cite{chen2022RAMPVIS}, our main objective is to provide VIS researchers and practitioners with valuable insights gained from our research and development (R\&D) activities conducted during the COVID-19 pandemic.
In an effort to predict the potential impact of diverse interventions, modelers have actively utilized COVID-19 data, employing a method known as Uncertainty Quantification (UQ).
This process seeks to measure uncertainties through the application of mathematical models and simulations.
However, modelers are faced with significant challenges, including the aspects of expert elicitation and effective communication.
In other words, there is a need for software engineering effort coupled with visualization to provide support for the validation and verification tests, and to create efficient workflows between modelers and researchers from other disciplines \cite{ackland2022Royal}.

\begin{figure}[tb!]
    \centering
    \includegraphics[width=0.65\linewidth]{venn.png}
    \caption{The organization of researchers from the SCRC and RAMPVis. The SCRC modeling team is responsible for developing the epidemiological models leveraging different modeling techniques. The RAMPVis team provides visualization support to the SCRC modeling team, by establishing four VIS volunteer teams who work on the actual development under the guidance of the RAMPVis team.
    }
    \label{fig:venn}

\end{figure}

In addressing these hurdles, \ac{VIS} emerge as a potent tool, offering the capacity to significantly enhance and streamline their collaborative workflows \cite{swallow2022Challenges}.
While our work may not have showcased the state-of-the-art VIS techniques, it effectively delivered rapid and practical VIS support to the modelers during an exceptional and demanding time.

Our contribution is an early history of our volunteer response from a software engineering and visualization perspective.
We present the earliest stages of the visualization dashboard, EnsembleDashVis, developed during the pandemic, aiming to assist the modelers in interpreting an \ac{ABC-SMC} inference model that they have developed using COVID-19 data collected during the first wave of the outbreak in Scotland \cite{scrc2020Covid19}. 
Much of this effort and the reasoning behind this volunteer work was never documented.

% revision
\noindent
\textbf{Unconventional Software Development:}  
The visualization software we developed in this project was developed under unconventional and unprecedented circumstances.
Some of the unusual properties of this software project were the amount of uncertainty as the project started.
The following aspects were unknown at the project start:
\begin{itemize}[itemsep=0pt,topsep=0pt]
    \item An unknown a priori requirements specification: We did not know what the user requirements and expectations were.
    \item An unknown project team: The members of the project team were unknown.
    We only knew the leader of the visualization team, Prof Min Chen. 
    In addition, the project team was dynamic with new members joining throughout.
    \item Unknown data characteristics: We did not know what the simulation data was at the start of the project.
    \item An unfamiliar work environment: The collective work environment landscape shifted to a work-at-home model which was new to the team at the time.
\end{itemize}
While arguably, these characteristics could describe other software engineering projects, we believe that the uncertainty in this particular case was unusually high.
All aspects of this project had the feel of laying down the tracks as the train was running.