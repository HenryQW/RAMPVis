\section{Introduction and Motivation}

The Scottish COVID-19 Response Consortium (SCRC) \cite{2020University}, in collaboration with the Royal Society's call to action, has taken a proactive approach in addressing the need for enhanced epidemiological models of COVID-19 transmission.
This joint effort, known as Rapid Assistance in Modelling the Pandemic (RAMP) \cite{2020Rapid}, aims to foster a deeper comprehension of the consequences associated with various exit strategies from lockdown measures.
Moreover, this consortium has attracted the involvement of distinguished scientists and experts from diverse organizations both within the United Kingdom and abroad, thus augmenting the collective knowledge base and ensuring comprehensive expertise in specialized domains.

RAMPVis \cite{2020Visualization} is a group of volunteers specialized in Data Visualization and Visual Analytics (abbreviated as VIS).
This group has willingly come forward to lend their specialized skills and knowledge in order to provide invaluable support to the consortium's modeling scientists and epidemiologists.
By harnessing the power of data visualization, RAMPVis aims to enhance the understanding and communication of complex COVID-19-related information.
Their ultimate goal is to facilitate informed decision-making processes.
Their collaboration with the SCRC exemplifies the spirit of interdisciplinary cooperation and collective efforts in combating the pandemic.

As the team responsible for the visualization component of one of the six epidemiological models developed by the SCRC modellers \cite{chen2022RAMPVIS}, we aim to provide VIS researchers and practitioners with valuable insights into our experience, knowledge, and reflections on our research and development (R\&D) activities carried out during the COVID-19 pandemic.
By examining our experiences, we aim to shed light on how VIS research and techniques have contributed to pandemic responses and identify areas for improvement to enhance support for future emergency responses.