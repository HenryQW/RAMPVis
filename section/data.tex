\section{Data Description}
\label{sec:Data}

\begin{table}[tb!]
    \centering
    \caption{16 input parameters for the ABC-SMC inference model. The constant $K$ here is ignored in our visual designs.}
    \label{tab:input_params}
    \resizebox{\columnwidth}{!}{%
        \begin{tabular}{|l|l|}
            \hline
            \textbf{Name} & \textbf{Description}                                                       \\
            \hline
            \hline
            T\_lat        & Mean latent period (days)                                                  \\
            \hline
            juvp\_s       & Probability of juvenile developing symptoms                                \\
            \hline
            T\_inf        & Mean asymptomatic period (days)                                            \\
            \hline
            T\_rec        & Mean time to recovery if symptomatic (days)                                \\
            \hline
            T\_sym        & Mean symptomatic period prior to hospitalization (days)                    \\
            \hline
            T\_hos        & Mean hospitalization stay (days)                                           \\
            \hline
            inf\_asym     & Reduction factor of infectiousness for asymptomatic infectious individuals \\
            \hline
            p\_inf        & Probability of Infection                                                   \\
            \hline
            p\_hcw        & Probability of Infection (Healthcare Worker)                               \\
            \hline
            c\_hcw        & Mean number of Healthcare Worker contacts per day                          \\
            \hline
            d             & Proportion of population observing social distancing                       \\
            \hline
            q             & Proportion of normal contact made by people self-isolating                 \\
            \hline
            p\_s          & Age-dependent probability of developing symptoms                           \\
            \hline
            rrd           & Risk of death if not hospitalized                                          \\
            \hline
            lambda        & Background transmission rate                                               \\
            \hline
            K             & Hospital bed capacity                                                      \\
            \hline
        \end{tabular}
    }
\end{table}

\begin{table}[tb!]
    \centering
    \caption{13 output parameters from the simulation performed by the ABC-SMC inference model.}
    \label{tab:output_param}
    \resizebox{\columnwidth}{!}{%
        \begin{tabular}{|l|p{9.25cm}|}
            \hline
            \textbf{Name} & \textbf{Description}
            \\
            \hline
            \hline

            iter          & The simulation number.
            \\
            \hline
            day           & The day number.
            \\
            \hline
            age\_group    & The age group of the population.
            \\
            \hline
            S             & Number of susceptible individuals (not infected).                                                           \\
            \hline
            E             & Number of infected individuals but not yet infectious (exposed).                                            \\
            \hline
            E\_t          & Number of exposed individuals and tested positive.                                                          \\
            \hline
            I\_p          & Number of infected and infectious symptomatic individuals but at pre-clinical stage (show yet no symptoms). \\
            \hline
            I\_t          & Number of tested positive individuals that are infectious.                                                  \\
            \hline
            I            & Number of infected and infectious asymptomatic individuals.                                    \\
            \hline
            I\_s         & Number of infected and infectious symptomatic individuals.                                     \\
            \hline
            H             & Number of infected individuals that are hospitalized.                                                       \\
            \hline
            R             & Number of infected individuals that have recovered from the infection.                                      \\
            \hline
            D             & Number of deceased individuals due to the disease.                                                          \\
            \hline
        \end{tabular}%
    }
\end{table}


The data used in our work includes simulation parameters and outcomes from an \ac{ABC-SMC} inference model \cite{toni2008Approximate} developed by a group of modelers from Durham University, the University of Edinburgh, the University of Exeter, the University of Glasgow, and the London School of Hygiene \& Tropical Medicine.
The pandemic data used for the simulation was collected by NHS Scotland during the first wave of the outbreak in Scotland spanning a period of 59 days \cite{2020Covid19}.
The model was built to analyze the data and infer the parameters of the model that best fit the data.
The model has 16 input parameters (see \Cref{tab:input_params}) and 13 output parameters (see \Cref{tab:output_param}), and it performed 1,000 simulations with 160 sets of input parameters that were inferred via random seeding.
The resulting output file is a CSV file that is over 6GB in size.

It is worth mentioning that after plotting the output data using a line chart, an error was immediately spotted, see \Cref{fig:1st-line}, where an unusual spike can be observed on day 20.
The modelers were notified and the bug was fixed.
However, the rectified output file was never made available to us.
