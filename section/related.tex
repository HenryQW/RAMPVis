\section{Background and Related Work}

\ac{VIS} has been widely utilized in critical applications such as emergency responses and healthcare, assisting public officials and decision-makers in understanding intricate datasets and extracting useful, actionable insights from them \cite{dusse2016Information}. 
\ac{VIS} has also played a prominent role in disseminating COVID-19 information through various media channels, it has contributed significantly to facilitating more efficient and clearer public communication, facilitating a broader understanding of the crisis \cite{johnshopkinsuniversityCOVID19}.

In our work, our primary objective was to extend support through \ac{VIS} to two distinct user groups. Firstly, the modelers, who could significantly benefit from VIS in comprehending their models more effectively and fine-tuning them accordingly. Secondly, to the epidemiologists, whom VIS could assist in interpreting the outcomes of these computational models. Our work is included in multiple publications \cite{chen2022RAMPVIS,dykes2022Visualizationb,khan2022Propagating,khan2022Rapid,rydow2023RAMPVIS}, where it functioned as the preliminary VIS prototype, shaping a portion of their respective studies.

\bobgraph{VIS for Emergency Response}

Previously, as co-authors we have detailed the related work focusing on the use of \ac{VIS} in emergency response, refer to the related work section in Chen et al. \cite{chen2022RAMPVIS}. The aforementioned literature review laid the foundation and was conducted prior to the development of our current study in 2020.

Maciejewski et al. \cite{maciejewski2011Pandemica} develop a VIS toolkit to analyze the effect of decision measures enforced during a simulated pandemic, the tool was later utilized by the Indiana State Department of Health during an outbreak of H1N1 (swine flu).
Ribicic et al. \cite{ribicic2012Sketching} leverage VIS with the intention of delivering real-time feedback derived from flood simulations to non-expert users, while Konev et al. \cite{konev2014Run} use VIS to support decision-making in flooding scenarios.

Jeitler et al. \cite{jeitler2019RescueMark} use VIS to analyze social media data to aid rescue teams, specifically in terms of optimal allocation of resources during emergency response situations. Similarly, Nguyen and Dang \cite{nguyen2019EQSA} harness social media data, paired with VIS, to facilitate and improve post-earthquake resource allocation and rescue effort.

In contrast to the majority of previous studies mentioned here that generally focus on preparing for future emergencies, our work was undertaken during the COVID-19 pandemic as a rapid response to an ongoing emergency.

\bobgraph{VIS for COVID-19 Data Modeling}

In the rest of the section, we focus on the use of \ac{VIS} in aiding the computational modeling of COVID-19 data. These studies were not published or available to us during the development of our work. In fact, the use of VIS in epidemiological modeling was rare, both modelers and epidemiologists might be unaware that they had such a potent instrument readily available \cite{chen2022RAMPVIS}.

He et al. \cite{he2020SEIR} developed an SEIR (Susceptible, Exposed, Infected, and Recovered) model for spread prediction by leveraging COVID-19 data obtained from the Hubei province in China. They employed a variety of 2D plots to estimate the parameters of the model and interpret the results that the model yielded. Godio et al. \cite{godio2020SEIR} took the same approach in developing an SEIR model for the Lombardy region in Italy.

The IHME COVID-19 Forecasting Team \cite{ihmecovid-19forecastingteam2021Modeling} take the application of data visualization (VIS) a step further in their development of the SEIR model for accessing social distance mandates, they extend the use of \ac{VIS} to include choropleth and violin plots, and small multiples for 2D plots.

Chinazzi et al. \cite{chinazzi2020Effect} develop a model to simulate the effectiveness of international travel restrictions in containing the spread of COVID-19. In addition to the use of 2D plots to refine their models, they also utilize a range of geospatial visualizations. This allowed them to more effectively interpret the results generated by their models. The use of geospatial visualizations is also adopted by Alvarez Castro and Ford \cite{alvarezcastro20213D} in their development of a model for analyzing transmission in a university campus in the UK.

In contrast to these studies that highlight the efficacy of \ac{VIS} in supporting the computational modeling of COVID-19 data with a primary focus on the model development, as they are formulated by the modelers, our study takes a different approach. We focus our attention on refining \ac{VIS} as a potent tool that can significantly improve the computational modeling of COVID-19 data, all viewed through the unique lens of a \ac{VIS} practitioner.
