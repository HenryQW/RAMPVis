\section{Background and Related Work}

\ac{VIS} has been widely utilized in critical applications and healthcare, including during the COVID-19 pandemic.
Data visualization has played a prominent role in disseminating COVID-19 information through various media channels.
This surge in \ac{VIS} adoption has facilitated a better understanding of the impact of the pandemic, helped identify trends and patterns, and empowered decision-makers with actionable insights derived from complex datasets.

Add a paragraph of related papers from the special issue in which Min Chen's paper is published.  Cite these papers.
https://www.sciencedirect.com/journal/epidemics/special-issue/10DM7ZPJKM9

Cite these papers in the special issue:
https://www.sciencedirect.com/journal/epidemics/special-issue/10DM7ZPJKM9

Our work is included in multiple publications \cite{chen2022RAMPVIS,dykes2022Visualizationb,khan2022Propagating,rydow2023RAMPVIS},  where it functioned as the preliminary VIS prototype, shaping a portion of their respective implementations.
As we have already described the related work in \ac{VIS} in emergency response and healthcare behind our implementation (See related work section in Chen et al. \cite{chen2022RAMPVIS}), in this section, we will provide a concise exploration of the various applications of \ac{VIS} in these domains, which served as a catalyst for our own development efforts.

In contrast to the majority of previous studies that typically focus on preparing for future emergencies, our work was undertaken during the COVID-19 pandemic as a rapid response to an ongoing emergency.